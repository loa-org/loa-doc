\chapter{Introduction}

Loa is a project that was started to address a set of specific reqiurements by combining two distinct technologies, microcontrollers and field pro\-gramm\-able gate arrays (FPGAs).

The intended application is a central controller for a Eurobot-competition type robot. This application requires a high number of interfaces, that are often timing critical and 
are usually impossible to implement in a centralized archtecture by using only a single microcotnroller. Reconfigurability of the FPGA is also welcome additon as this allows 
for incereased flexibilty in the rather short development cycles dictated by the competition context.

High-Level design objectives are:

\begin{itemize}
\item Simplicity 
\item Accept a narrow design scope for short and medium term benefits
\item Pragmatic usability - Allow for rapid prototyping
\item Open Source where reasonably possible
\end{itemize}



\section{System Design Decisions}

FPGA:
\begin{itemize}
\item Simple FPGA design are easier to verify, therefore Loa uses only one clock domain, no reset, is strictly synchronous.
\item Use of \href{http://www.gaisler.com/doc/vhdl2proc.pdf}{2-process statemachines} to ease reading and understanding of the code.
\item Embrace limitation of hardware, no need to over-engineer the system. 
\end{itemize}

Hardware:
\begin{itemize}
\item Cortex-M4 was choosen over larger system for its deterministic timing behavior at still high computational performance.
\item FPGA choice was limited to packages that can be hand soldered and inspected visually.
\item 100 mil and 2.0 mm headers have been choosen over high-density board-to-board connector to ease assembly, integration and prototyping.
\item Low power consumption and high-speed I/Os are considered secondary, as these are not required in the robots.
\end{itemize}


Software:
\begin{itemize}
\item To interface well with existing \href{http://www.roboterclub.rwth-aachen.de}{RCA} software the XPCC framework is used.
\item ANY MORE?
\end{itemize}




